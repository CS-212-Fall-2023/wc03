\documentclass[a4paper]{exam}

\usepackage{amsmath,amssymb, amsthm}
\usepackage[a4paper]{geometry}
\usepackage{hyperref}
\usepackage{mdframed}

\title{Weekly Challenge 03: Closure of Regular Languages}
\author{CS 212 Nature of Computation\\Habib University}
\date{Fall 2023}

\qformat{{\large\bf \thequestion. \thequestiontitle}\hfill}
\boxedpoints

\usepackage{draftwatermark}
\SetWatermarkText{Sample Solution}
\SetWatermarkScale{3}
\printanswers

\begin{document}
\maketitle

\begin{questions}
  
\titledquestion{Operation F}

  Let us define a unary operation, $F$, on languages as follows.
  
  \begin{mdframed}
    $F(L) = \{ f(w) \mid w\in L\}$ where, given
    \begin{itemize}
    \item $w=w_1w_2w_3\ldots w_n$, and
    \item each $w_i\in\Sigma$,
    \end{itemize}
    $f(w) = w_nw_{n-1}w_{n-2}\ldots w_1$.
  \end{mdframed}

  Prove or disprove that the class of regular languages is closed under $F$.

  % Uncomment below to enter your solution.
  \begin{solution}
    We have to show that if $L$ is regular, then so is $F(L)$. We observe that $f(w)$ is the reverse string of $w$. So, $F(L)$ contains strings that are the reverse of the strings in $L$. It is also useful to note that $w=f(f(w))$.
    
    \begin{proof}
      We provide a proof by construction. That is, starting with the NFA, $N_1$, that accepts $L$, we build an NFA, $N_2$, that accepts $F(L)$. We do so by inverting the transitions in $N_1$, and swapping the status of its start and final states. A mechanism will be needed in $N_2$ for multiple start states, and care is required for the case when the start state in $N_1$ is also a final state.\\
      
      Let $N_1 = (Q,\Sigma_\epsilon,\delta_1,q_1, F_1)$.\\
      Define $N_2 = (Q\cup \{q_2\},\Sigma_\epsilon,\delta_2,q_2, F_2)$ such that:
      \begin{enumerate}
      \item $F_2=\{ q_1\}$.\\
        \textit{the start state of $N_1$ is the final state of $N_2$.}
      \item $\forall q\in F_1\; (\delta_2(q_2,\epsilon) = q)$.\\
        \textit{$q_2$ is an additional state in $N_2$. It is the start state with $\epsilon$-transitions to the states that were final in $N_1$.}
      \item $\forall q_1q_2\in F_1\forall a \in \Sigma_\epsilon\; (q_2\in \delta_1(q_1,a) \implies q_1\in\delta_2(q_2,a))$.\\
        \textit{invert the transitions of $N_1$ in $N_2$.}
      \end{enumerate}
    \end{proof}

    We may also want to prove the correctness of this construction.
    \begin{proof}
      Let $L_2$ be the language of $N_2$. We show that $L_2 = F(L)$. That is,
      \[
        w\in L_2\implies w\in F(L)\quad \land\quad  w\in F(L)\implies w\in L_2.
      \]
      We sketch the proof of each case below.

      \underline{Case 1}: $w\in L_2\implies w\in F(L)$\\
      Let $w=w_1w_2\ldots w_n$.\\
      $N_2$ traverses a sequence of states, $\langle r_0,r_1,r_2,\ldots, r_n\rangle $, as it reads $s$ where
      \begin{itemize}
      \item $r_0 = q_2$,
      \item $r_1\in F_1$,
      \item $r_n = q_1$, and
      \item $\delta_2(r_i,w_{i+1}) = r_{i+1}$.
      \end{itemize}
      Then, $N_1$ traverses the states, $\langle q_1, r_{n-1}, r_{n-2}, \ldots, r_1\rangle$, for the string $w_nw_{n-1}\ldots w_2w_1= f(w)$.\\
      That is, $N_1$ accepts $f(w)$.\\
      $\therefore w\in F(L)$.
      
      \underline{Case 2}: $w\in F(L)\implies w\in L_2$\\
      This can be argued similarly to the above case.      
    \end{proof}
    
  \end{solution}
  
\end{questions}
\end{document}

%%% Local Variables:
%%% mode: latex
%%% TeX-master: t
%%% End:
